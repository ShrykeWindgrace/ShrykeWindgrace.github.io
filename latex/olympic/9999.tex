\documentclass[a4paper, 12pt]{article}
\usepackage{amsmath}
\newcommand\pmt{\pmod {10}}
\newcommand\pmtt{\pmod {100}}
\begin{document}
How many zeros does $9^{999}+1$ end with?

Essentially, we need to find the maximum power of $10$ that divides this number.
Since $9$ and $10$ are coprime, we can safely factor out terms that are not divisible by $10$ and lose nothing.

First of all
\[
9^{999} +1 = (9^{333}+1)(9^{666} - 9 ^{333} + 1).
\]
The second factor is not divisible by $10$, because, as $9\equiv-1 \pmt$, we write
\[
    9^{666} - 9 ^{333} + 1 \equiv (-1)^{666} - (-1)^{333} + 1 \pmt \equiv 3 \pmt.
\]
Therefore, it is sufficient to study $9^{333}+1$.
Once again
\[
9^{333} +1 = (9^{111}+1)(9^{222} - 9 ^{111} + 1),
\]
and for exactly the reason as before the second factor is not divisble by $10$, hence we study $9^{111}+1$.

And again
\[
9^{111} +1 = (9^{37}+1)(9^{74} - 9 ^{37} + 1),
\]
and we reduced the problem to the number $9^{37}+1$.
Clearly, this nimber is divisible by $10$:
\[
    9^{37}+1 \equiv (-1)^{37} + 1 \pmt \equiv 0 \pmt
\]
Now let's see whether this number is divisble by $100$; to do so, we study the following remainders:
\[ 9^2 \equiv 81 \pmtt\]
\[ 9^4 \equiv 81^2 \pmtt \equiv (6400 + 160 + 1) \pmtt \equiv 61 \pmtt\]
\[ 9^8 \equiv 61^2 \pmtt \equiv (3600 + 120 + 1) \pmtt \equiv 21 \pmtt\]
\[ 9^{16} \equiv 21^2 \pmtt \equiv (400 + 40 + 1) \pmtt \equiv 41 \pmtt\]
\[ 9^{32} \equiv 41^2 \pmtt \equiv (1600 + 80 + 1) \pmtt \equiv 81 \pmtt\]
Therefore, as $9^{37} = 9^{32} \times 9^4 \times 9$, we get
\[9^{37} \equiv (9^{32} \times 9^4 \times 9) \pmtt \equiv (81 \times 61 \times 9) \pmtt\]
\[
\equiv (41 \times 9) \pmtt \equiv 369 \pmtt \equiv 69 \pmtt
\]
And this leaves us with
\[9^{37}+1 \equiv 70\pmtt,\]
or, in other words, it is not divisible by $100$.
We finally conclude that $9^{999}+1$ ends with only \textbf{one} zero.
\end{document}