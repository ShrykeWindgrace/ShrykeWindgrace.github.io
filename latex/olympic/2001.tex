\documentclass[12pt, a4paper]{article}
\usepackage[utf8]{inputenc}
\usepackage[T2A]{fontenc}
\usepackage{amsmath}
\usepackage[russian]{babel}
\newcommand\pmn{\pmod {9}}
\begin{document}
Все натуральные числа от 1 до 2001 выписаны одно за другим, получилось одно большое число.
Делится ли оно на 9?


На самой олимпиаде я получил правильный ответ (нет, не делится), я муторным способом пытался посчитать сумму цифр этого числа.
Эта сумма на 9 не делилась, но остаток был неправильный.
Не знаю, сколько мне за эту задачу дали баллов.


Приписать число $b$ к числу $a$ и получить $\overline{ab}$ - это всё равно что посчитать сумму $10^ka+b$ для некоторого положительного $k$
(оно равно количеству цифр в десятичной записи числа $b$).
Также, поскольку $10\equiv1\pmn$ и $10^k\equiv1\pmn$, сразу же можно получить соотношение
\[
\overline{ab}\equiv (10^ka+ b)\pmn \equiv a+b \pmn.
\]
После этого легко записать
\[
\overline{1\ldots 2001} \equiv \left(\sum_{k=1}^{2001}k\right) \pmn \equiv  2001\times 1001 \pmn 
\]
2001 делится на 3, но не делится на 9, а 1001 вообще не делится на 3, значит и их произведение не делится на 9.

Отсюда вывод - описанное большое число на 9 не делится.

\end{document}
