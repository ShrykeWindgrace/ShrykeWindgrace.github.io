\documentclass[12pt, a4paper]{article}
\usepackage{amsmath}
\usepackage[
    backend=biber,
    style=alphabetic,
]{biblatex}
\usepackage{url}
\bibliography{ref}
\begin{document}
\noindent Consider a polynomial $x^{33}-1$.
Its roots are
\[
    e^{\frac{2\pi ki}{33}},\quad k=0,\ldots,32.
\]
We can factorize this polynomial using cyclotomic polynomials (cf. wikipedia article \cite{cp}):
\[
    x^{33}-1 = \Phi_1(x) \Phi_3(x) \Phi_{11}(x)\Phi_{33}(x)
\]
where
\[
    \Phi_n(x) = \prod_{1\le k\le n,\, gcd(k,n)=1}\left(x-e^{\frac{2\pi k i}{n}}\right).
\]
One of useful properties of these cyclotomic polynomials: if $p$ is prime, then
\[
    \Phi_p(x) = \frac{x^p-1}{x-1}.
\]
We are mostly interested in the $\Phi_{33}(x)$.
The indices $k$ coprime with $33$ form a set
\[
    I = \{1, 2, 4, 5, 7, 8, 10, 13, 14, 16, 17, 19, 20, 23, 25, 26, 28, 29, 31, 32\}.
\]
One could also notice that
\[
    e^{\frac{2\pi k i}{n}} = \overline{ e^{\frac{2\pi (n - k)i}{n}}},
\]
and when we regroup the indices in $I$  as $(1, 32)$, $(2,31)$ and so for, we can write
\[
    \Phi_{33}(x) = \prod_{k\in I} \left(x-e^{\frac{2\pi k i}{33}}\right)
    =\prod_{k\in I'} \left(x^2-2x\cos\left( {\frac{2\pi k}{33}}\right)+1\right)
\]
with
\[
    I' = \{1, 2, 4, 5, 7, 8, 10, 13, 14, 16\}.
\]
Now let's plug $x=i$ in the polynomial $x^{33}-1$:
\[
    i^{33}-1 = \Phi_1(i) \Phi_3(i) \Phi_{11}(i)\Phi_{33}(i) = (i-1)\frac{i^3-1}{i-1} \frac{i^{11}-1}{i-1}\Phi_{33}(i).
\]
After simplifying, we get
\[
    \Phi_{33}(i) = -1.
\]
On the other hand, the expression in terms of cosines yields
\[
    \Phi_{33}(i) = \prod_{k\in I'}\left(-2i\cos \left( {\frac{2\pi k}{33}}\right) \right) = -2^{10}
    \prod_{k\in I'}\left(\cos \left( {\frac{2\pi k}{33}}\right) \right)
\]
\begin{align*}
    = -&2^{10}\cos \left( {\frac{2\pi}{33}}\right)
    \cos \left( {\frac{4\pi }{33}}\right)
    \cos \left( {\frac{8\pi }{33}}\right)
    \cos \left( {\frac{10\pi }{33}}\right)
    \cos \left( {\frac{14\pi }{33}}\right)
    \cos \left( {\frac{16\pi }{33}}\right)
    \\&\times
    \cos \left( {\frac{20\pi }{33}}\right)
    \cos \left( {\frac{26\pi }{33}}\right)
    \cos \left( {\frac{28\pi }{33}}\right)
    \cos \left( {\frac{32\pi }{33}}\right)
\end{align*}
\begin{align*}
    = -&2^{10}\cos \left( {\frac{2\pi}{33}}\right)
    \cos \left( {\frac{4\pi }{33}}\right)
    \cos \left( {\frac{8\pi }{33}}\right)
    \cos \left( {\frac{10\pi }{33}}\right)
    \cos \left( {\frac{14\pi }{33}}\right)
    \cos \left( {\frac{16\pi }{33}}\right)
    \\&\times
    \cos \left( {\frac{13\pi }{33}}\right)
    \cos \left( {\frac{7\pi }{33}}\right)
    \cos \left( {\frac{5\pi }{33}}\right)
    \cos \left( {\frac{\pi }{33}}\right)
\end{align*}
Now put the value of $\Phi_{33}(i)$, multiply both sides by $\sin \left(\frac{\pi}{33}\right)$
and reorder the factors:
\begin{align*}
    \sin \left(\frac{\pi}{33}\right) =&\, 2^{10}
    \cos \left( {\frac{5\pi }{33}}\right)
    \cos \left( {\frac{7\pi }{33}}\right)
    \cos \left( {\frac{10\pi }{33}}\right)
    \cos \left( {\frac{13\pi }{33}}\right)
    \cos \left( {\frac{14\pi }{33}}\right)\\
    &\times
    \cos \left( {\frac{16\pi }{33}}\right)
    \cos \left( {\frac{8\pi }{33}}\right)
    \cos \left( {\frac{4\pi }{33}}\right)
    \cos \left( {\frac{2\pi}{33}}\right)
    \cos \left( {\frac{\pi }{33}}\right)
    \sin \left( {\frac{\pi }{33}}\right).
\end{align*}
Notice the formula for the double angle in the last product:
\begin{align*}
    = & \, 2^{9}
    \cos \left( {\frac{5\pi }{33}}\right)
    \cos \left( {\frac{7\pi }{33}}\right)
    \cos \left( {\frac{10\pi }{33}}\right)
    \cos \left( {\frac{13\pi }{33}}\right)
    \cos \left( {\frac{14\pi }{33}}\right)
    \\&\times
    \cos \left( {\frac{16\pi }{33}}\right)
    \cos \left( {\frac{8\pi }{33}}\right)
    \cos \left( {\frac{4\pi }{33}}\right)
    \cos \left( {\frac{2\pi}{33}}\right)
    \sin \left( {\frac{2\pi }{33}}\right),
\end{align*}
rinse and repeat:
\begin{align*}
    = &\, 2^{8}
    \cos \left( {\frac{5\pi }{33}}\right)
    \cos \left( {\frac{7\pi }{33}}\right)
    \cos \left( {\frac{10\pi }{33}}\right)
    \cos \left( {\frac{13\pi }{33}}\right)
    \cos \left( {\frac{14\pi }{33}}\right)
    \\&\times
    \cos \left( {\frac{16\pi }{33}}\right)
    \cos \left( {\frac{8\pi }{33}}\right)
    \cos \left( {\frac{4\pi }{33}}\right)
    \sin \left( {\frac{4\pi }{33}}\right)
\end{align*}
\begin{align*}
    = &\, 2^{7}
    \cos \left( {\frac{5\pi }{33}}\right)
    \cos \left( {\frac{7\pi }{33}}\right)
    \cos \left( {\frac{10\pi }{33}}\right)
    \cos \left( {\frac{13\pi }{33}}\right)
    \cos \left( {\frac{14\pi }{33}}\right)
    \\&\times
    \cos \left( {\frac{16\pi }{33}}\right)
    \cos \left( {\frac{8\pi }{33}}\right)
    \sin \left( {\frac{8\pi }{33}}\right)
\end{align*}
\begin{align*}
    = &\, 2^{6}
    \cos \left( {\frac{5\pi }{33}}\right)
    \cos \left( {\frac{7\pi }{33}}\right)
    \cos \left( {\frac{10\pi }{33}}\right)
    \cos \left( {\frac{13\pi }{33}}\right)
    \cos \left( {\frac{14\pi }{33}}\right)
    \\&\times
    \cos \left( {\frac{16\pi }{33}}\right)
    \sin \left( {\frac{16\pi }{33}}\right)
\end{align*}
\[
    = 2^{5}
    \cos \left( {\frac{5\pi }{33}}\right)
    \cos \left( {\frac{7\pi }{33}}\right)
    \cos \left( {\frac{10\pi }{33}}\right)
    \cos \left( {\frac{13\pi }{33}}\right)
    \cos \left( {\frac{14\pi }{33}}\right)
    \sin \left( {\frac{32\pi }{33}}\right).
\]
Finally, $\sin \left( {\frac{32\pi }{33}}\right) = \sin \left( {\frac{\pi }{33}}\right)$, and we conclude that
\[
    \cos \left( {\frac{5\pi }{33}}\right)
    \cos \left( {\frac{7\pi }{33}}\right)
    \cos \left( {\frac{10\pi }{33}}\right)
    \cos \left( {\frac{13\pi }{33}}\right)
    \cos \left( {\frac{14\pi }{33}}\right)
    =\frac {1}{32}.
\]
\printbibliography
\end{document}
